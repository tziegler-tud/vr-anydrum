\documentclass[11pt,twocolumn,twoside,lineno]{pnas-new}
% Use the lineno option to display guide line numbers if required.
% Note that the use of elements such as single-column equations
% may affect the guide line number alignment. 

\templatetype{pnasmathematics} % Choose template 
% {pnasresearcharticle} = Template for a two-column research article
% {pnasmathematics} = Template for a one-column mathematics article
% {pnasinvited} = Template for a PNAS invited submission

\usepackage{pdfpages}
\usepackage{booktabs}
\usepackage{tabularx}
\usepackage{graphicx}
\usepackage{subcaption}
\usepackage{multirow}
\usepackage[square]{natbib}
%\usepackage[nohyperlinks, printonlyused]{acronym}

%%enable Paragraphs
\usepackage{pgfgantt}
\usepackage{parskip}
%\usepackage{babel}
%\usepackage{translator}
\usepackage{pdfpages}
 \usepackage{pdflscape}
%\usepackage[hidelinks]{hyperref}

\title{EveryDrum}

% Use letters for affiliations, numbers to show equal authorship (if applicable) and to indicate the corresponding author
\author[1]{Autoren:Tom Ziegler}
\author[1]{Lukas Peschel} 
\author[1]{\\ Betreuer: Dr. Sebastian Merchel}

\affil[1]{Technische Universität Dresden}

\keywords{Drums $|$ Android $|$ Arduino $|$ PureData $|$} 

\begin{abstract}
Das bestimmen von Erschütterungen an Objekten kann mit Hilfe verschiedener Sensoren geschehen.
Dabei können Piezzosensoren oder Beschleunigungssensoren zum Einsatz kommen.
Um Erschütterungen auf einer Ebene zu lokalisieren und voneinander von ihrer Stärke zu unterscheiden, wurden Beschleunigungssensoren verwendet.

Verwendet wurde der MPU-6050 Sensor zusammen mit einem Arduino Nano.
Die Auswertung der Sensordaten fand auf einem Android-Smartphone statt.
Um die Erkennung von Erschütterungen zu testen, wurden einzelne Schläge auf einen Tisch ausgewertet.
Mehrere Schläge wurden analysiert und kategorisiert, indem ihnen einzelne Schlagzeuggeräusche zugeordnet wurden.
Die Analyse wurde mit Hilfe einer FFT-Funktion durchgeführt.

Einzelne Schläge liesen sich damit gut bestimmen mit einem Sensor.
Ebenso zwei Schläge an unterschiedlichen Positionen konnten mit geringer Fehlerrate unterschieden werden.

\end{abstract}

\dates{Erstellt am \myformat\today}
%\doi{\url{www.pnas.org/cgi/doi/10.1073/pnas.XXXXXXXXXX}}

\begin{document}

% Optional adjustment to line up main text (after abstract) of first page with line numbers, when using both lineno and twocolumn options.
% You should only change this length when you've finalised the article contents.
\verticaladjustment{-2pt}

\maketitle
\thispagestyle{firststyle}
\ifthenelse{\boolean{shortarticle}}{\ifthenelse{\boolean{singlecolumn}}{\abscontentformatted}{\abscontent}}{}

\section{Einleitung}
Mit Hilfe von Piezosensoren, Kraftsensoren, soll es möglich sein von beliebigen Oberflächen Schwingungen aufzunehmen, die beim Klopfen auf das Objekt entstehen.

Das Klopfen soll abhängig von der Position auf dem Objekt und der Stärke einen unterschiedlichen DrumSound erzeugen. 
Zu den Möglichen Geräuschen zählen Trommeln, Basstrommel und Zimbal.

Die Trommelgeräusche werden mittels PureData synthetisiert. 
Die Geräusche werden in Abhängigkeit von der Entfernung und Stärke zum Piezosensor ausgewählt.
Eine Mobile-App wertet die Daten des Piezosensors aus und sppielt das entsprechende Geräusch ab.


\subsection*{Fragestellungen}
\begin{itemize}
	\item Welche Materialien/Objekte eignen sich am Besten für AnyDrum?
	\item Wie genau lässt sich die Frequenz beim Tippen ermitteln?
	\item Wie gut lassen sich schwache Schläge von Starken unterscheiden?
	\item Wie gut ist die Lokalisationserkennung zwischen einem Piezo, zwei, drei und vier?
\end{itemize}
%  \section*{Zetiplan}
 \begin{landscape}
 	
\newgantttimeslotformat{stardate}{
	\def\decomposestardate##1.##2\relax{\def\stardateyear{##1}\def\stardateday{##2}
	}
	
	\decomposestardate#1\relax
	\pgfcalendardatetojulian{
	\stardateyear-01-01}{#2}
	\advance#2 by-1\relax
	\advance#2 by\stardateday\relax
	}
%April
\begin{ganttchart}[hgrid,vgrid,time slot format=stardate, inline,
	milestone inline label node/.append style={left=5mm}]
	{2018.99}{2018.120}
	\gantttitlecalendar{, month=name, day}\\
	\ganttgroup{Einführung}{2018.99}{2018.106}\\
	\ganttbar{Signalerfassung und DrumSounds}{2018.107}{2018.120}
\end{ganttchart}
%Mai
\begin{ganttchart}[hgrid,vgrid,time slot format=stardate, inline,
	milestone inline label node/.append style={left=5mm}]
	{2018.121}{2018.151}
	\gantttitlecalendar{, month=name, week} \\
	\ganttbar{Prototyp Singnaleinbindung}{2018.121}{2018.134}\\
	\ganttbar{Prototyp Testen}{2018.136}{2018.145}\\
	\ganttmilestone{Zwischenpräsentation}{2018.146} \\
	\ganttbar{Verbessern}{2018.147}{2018.151}
	\ganttlink{elem0}{elem1}
	\ganttlink{elem1}{elem2}
	\ganttlink{elem2}{elem3}	
\end{ganttchart}
%Juni - Juli
\begin{ganttchart}[hgrid,vgrid,time slot format=stardate, inline,
	milestone inline label node/.append style={left=5mm}]
	{2018.152}{2018.195}
	\gantttitlecalendar{, month=name, week}\\
	\ganttbar{Erweitern \& verfeinern}{2018.152}{2018.171}\\
	\ganttbar{Testen}{2018.172}{2018.179}\\
	\ganttbar{Präsentation vorbereiten}{2018.175}{2018.182}\\
	\ganttbar{Puffer}{2018.182}{2018.188}\\
	\ganttmilestone{Endpräsentation}{2018.189} 
	\ganttlink{elem0}{elem1}
	\ganttlink{elem1}{elem2}	

	\ganttlink{elem2}{elem3}
	
	\ganttlink{elem3}{elem4}
\end{ganttchart}

\end{landscape}


\section{Konzept}
\label{sec:Konzept}
Ein naiver Ansatz wäre, mehrere Sensoren auf der Oberfläche zu verteilen und eine ein- oder zweidimensionale Lokalisation mittels Bestimmung der Laufzeitunterschiede durchzuführen.
Dies setzt allerdings die Verwendung mindestens zweier (eindimensional, max. 3 Positionen) bzw. dreier (zweidimensional) Sensoren. 
Die Verwendung mehrerer Sensoren bedeutet jedoch auch steigende Anforderungen an die verarbeitende Hardware, höherer Realisierungsaufwand, höhere Anforderungen an den Anwender und nicht zuletzt höheren Hardwareaufwand und -abhängigkeit, was die Portabilität und Verwendbarbarkeit deutlich einschränkt.

Daher soll lediglich ein Sensor verwendet werden. Dies bedeutet jedoch, dass die Unterscheidung der Anschlagpositionen lediglich anhand der Schwingungscharakteristiken erfolgen muss.

Wird eine Tischplatte an einer Position angeschlagen, führt diese eine Eigenschwingung aus.
Je nach Material, Lagerung, Obejekte auf der Oberfläche, etc., unterscheiden sich verschiedene Tischplatten in ihrer Impulsantwort.
Außerdem unterscheidet sich die ausgeführte Schwingung je nach Position und Art der Erregung.
Ein Schlag mit den Fingerknochen erzeugt eine andere Schwingung, als ein Schlag mit der flachen Hand.
Dies lässt sich anhand des entstehenden Geräusches verifizieren.
Ebenso erzeugt ein Schlag auf die Tischkante eine andere Schwingung, als ein Schlag in die Tischmitte, welches sich vor allem in der unterschiedlichen Nachhallzeit zeigt.

Gewisse charakteristische Eigenschaften sind bauartbedingt durch den Tisch vorgegeben. 
Diese können vom Anwender nicht beeinflusst werden, folglich muss die Software in der Lage sein, diese Unterschiede in der Verarbeitung zu berücksichtigen. 
Trotzdem hat der Anwender gewisse Möglichkeiten, die Schwingung der Tischplatte zu beeinflussen, beispielsweise durch Anbringen von Massen, Dämpfern, etc.
Dies kann und soll bewusst eingesetzt werden, um die Schwingungscharakteristiken der Anschlagspositionen möglichst unterschiedlich zu gestalten.  

Um eine hohe Flexibilität in der Anwendung zu liefern, soll auf die Verwednung eines PCs/Laptops verzichtet werden. Stattdessen soll die softwareseitige Umsetzung mittels eines Smartphones erfolgen.

Zur Aufnahme der Schwingung wurden verschiedene Möglichkeiten erprobt:
\begin{itemize}
	\item Die Verwendung eines Piezzosensors zur Aufnahme von Schwingungen.
	\item Messen der erzeugten Beschleunigung, mit Hilfe eines internen Beschleunigungssensors des Smartphones
	\item Das Messen mittels externer Beschleunigungssensoren eines Arduino Mikrocontrollers  
\end{itemize}


\section{Umsetzung}
Für die Umsetzung wurde ein Android-Smartphone, ein Lenovo K6, mit der Android-Version 7.0 verwendet.
Hingegen zu älteren Smartphones unterstützen neuere den USB On-The-Go (OTG) Modus, womit die Stromversorgung des Arduinos über USB-Anschluss gewährleistet wird.

Desweiteren wurde ein Arduino CH340 chip verwendet und als Beschleunigungssensoren das Modell MPU-6050.
Die Android-App wurde in Java geschrieben.
Zur Kommunikation mit dem Arduino und abgreifen der Beschleunigungssensoren des Sensors wurden die Bibliotheken \textit{usb-serial-for-android} \cite{mik3y} und \textit{UsbSerial} \cite{felHR85} verwendet.

Die synthetisch erzeugten Schlagzeuggeräusche für Hihat, Snare und Bass wurden mit PureData \cite{puredata} erzeugt. 
Zur Nutzung der PureData-Patches auf dem Android Gerät, wurde die Bibliothek \textit{pd-for-android} \cite{pdAndroid} von \textit{libpd} verwendet.
Diese unterstützt keine erweiterten Funktionen, wie sie in PureData Extended vorkommen.


\section{Arduino}


\subsection{CH340}
Bei dem Arduino CH340 handelt es sich um einen BUS Konvertierungs Chip, der eine serielle USB Schnittstelle ansprechen kann.

\cite{CH340}
\subsection{ MPU-6050}
\section*{Signalanalyse}
\label{sec:Signalanalyse}
\subsection*{Anschlagserkennung}
Zur Erkennung von von Anschlägen und Abgrenzung von Hintergrundrauschen werden die Messdaten in Echtzeit geprüft. Hierzu werden die jeweils letzten beiden Messwerte miteinander verglichen. Ein Anschlag wird erkannt, wenn der aktuellere Wert den vorherigen Wert um ein bestimmtes Vielfaches der Standartabweichung übersteigt. Der Faktor sollte dabei auf das zu erwartende Rauschen abgestimmt werden - höhere Werte neigen weniger dazu, Ausschläge im Rauschen bzw. unabsichtliche Behrührungen der Oberfläsche als Anschlag zu erkennen, niedrigere Werte erlauben dagegen, auch sanfte Anschläge zu erkennen. In unseren Versuchen hat sich ein Faktor von 6 als stabil erwiesen.

Wurde ein Anschlag erkannt, so beginnt ein Zeitfenster von ca. 500ms, in dem eingehende Messwerte gepuffert werden. Nach Ablauf des Zeitfensters wird dem Programm signalisiert, dass ein Schlag aufgenommen wurde. Zusätzlich werden die Messwerte bis ca. 20ms vor dem erkannten Anschlag in die Zeitreihe aufgenommen.

\subsection*{Analyse der Beschleunigungsdaten}
Die Klassifizierung aufgenommener Schläge erfolgt mittels Auswertung der stärksten Frequenzen. Hierzu werden die Messwerte zunächst einer diskreten Fouriertransformation unterzogen, die Spektrale Leistungssdichte der Zeitreihe bestimmt.
In der aktuellen Implementierung konnten Abtastraten von ca. 200Hz erreicht werden. 
Für ein Zeitfenster von ca. 500ms ergeben sich somit 128 Messwerte. 
Nach Anwendung der Fouriertransformation erhält man somit 64 diskrete Frequenzgruppen mit einer Trennschärfe von ca. 2Hz.

Aus diesen Daten werden die 4 Frequenzanteile mit der höchsten Leistungsdichte ermittelt zur weiteren Verwendung gespeichert.

\subsection*{Lernphase}
Zum Erlernen einer Schlagposition werden 10 Schläge aufgezeichnet und der Durchschnitt der jeweils stärksten Frequenzgruppe, zweitstärksten Frequenzgruppe, usw... ermittelt. Diese werden als charakteristisches Pattern gespeichert.
Tabelle \ref{tab:FFT} zeigt die zwölf Schläge zum Lernen für Hihat und Bass.
Die Schläge wurden auf einer Tischplatte ausgeführt mit 

\begin{figure}[H]
	\centering
	\caption{Stärkste Frequenzen}
	\begin{tabular}{l c c c c | l c c c c}
		HiHat &&&& &Bass \\
		\hline
		Frequenz & F1 & F2 & F3 & F4  & & F1 & F2 & F3 & F4\\	
		& 13 & 14 & 15 & 19  &&13& 114& 115& 87\\
		& 30 & 31 & 23 & 24 &&13& 28& 29& 30\\
		& 13& 14& 15& 16 &&35& 36& 23& 24\\
		& 29& 30& 13&36 &&13& 14& 15& 62\\
		& 17& 23&24& 30 	&&15& 16& 42& 43\\
		&13& 24& 25& 47	&&13& 54& 30& 31\\

		&27& 15& 16& 17	&&20& 89& 90& 39\\

		 &17& 18& 30& 31 &&13& 20& 26& 27\\

		&42& 43& 44& 55		&&19& 20& 29& 30\\

		 &25& 33& 34& 35 &&19& 20& 13& 26 \\

		&22& 23& 13& 37		&&13& 14& 15& 16\\

		&31& 32& 14& 15  &&13& 58& 59& 60\\
		\\
		\hline
		Durchschnitt & 23,25&	25	&22,167& 30,167 && 16,583	&40,25&	40,5&	39,583	\\
		Median & 23,5 &	23,5& 19,5	& 30,5 && 13 &	24	& 29	 &30,5 \\		 
		

	\end{tabular}
	\label{tab:FFT}
\end{figure}

\begin{figure}[H]
\centering
\begin{subfigure}{.5\textwidth}
		\includegraphics[scale=0.5]{figures/Mittelwert_hihatbass.png}
\end{subfigure}
\caption{Mittelwerte von Hihat und Bass für Frequenzen F1 - F4}
\label{fig:FFT_Mittelwerte}
\end{figure}


\begin{figure}[H]
\centering
\begin{subfigure}{.5\textwidth}
		\includegraphics[scale=0.5]{figures/Median_hihatbass.png}
\end{subfigure}
\caption{Median-Werte  von Hihat und Bass für Frequenzen F1 - F4}
\label{fig:FFT}
\end{figure}


\subsection*{Echtzeit-Klassifizierung}
Ist die Echtzeit-Klassifizierung aktiviert, so werden nach erfolgter Anschlagserkennung und Schlagaufzeichnung die Messwerte analysiert und die stärksten Frequenzgruppen mit denen der gelernten Positionen verglichen. Der Schlag wird der Position mit der gerinsten totalen Abweichung zugeordnet.
Erkennbar hierbei ist, dass in jedem Falle eine Zuordnung erfolgt - auch wenn die Abweichung absurd groß ist. Dies könnte jedoch leicht geändert werden, indem eine maximale Abweichung festgelegt wird und alle berechneten Abweichungen, die diesen Wert übersteigen, als nicht klassifizierbar eingestuft werden. Alternativ könnte, ähnlich zu anderen Clustering-Verfahren, eine Nachbarschaftsbeziehung zur Klassifikation verwendet werden.

\subsection*{Signalaufbereitung/Filterung}

In der aktuellen Implementierung erfolgt keine Filterung oder Nachbereitung der Messwerte. 


Eine Möglichkeit, um Rauschen zu unterdrücken wäre die Nutzung eines Grenzwertes. Auf diese Art können schwache Erschütterungen und leichte Vibrationen unterdrückt werden.
Die Sensordaten könnten weiter geglättet werden durch Verwendung eines Box-Filters (Moving-Average) oder Gauss-Filters.

%Anstelle von timbreID, welches laut Entwickler \cite{timbreID} Cluster benutzt und Merkmale sortiert, haben wir den DBSCAN-Algorithmus getetest.

%Elbatta und Ashour untersuchen verschiedene Clustering Ansätze und stellen in ihrer Arbeit \cite{Elbatta2013ADM} einen verbesserten DBSCAN-Algorithmus vor.
%DBSCAN ist ein  dichtebasierter Cluster-Algorithmus, der als Parameter einen Radius und eine Mindestanzahl an Punkte pro Cluster benötigt.
%Im Gegensatz zu partitionierungsbasierten Cluster-Algorithmen, wie zum Beispiel K-Means, besitzt DBSCAN dne Vorteil, dass es freie Formen von Clustern erkennen kann und sich besonders gut für Daten mit %Rauschen eignet.
%Der Algorithmus startet mit einem zufälligen Punkt.
%Befinden sich die Mindestanzahl an Nachbarpunkten um den gewählten Punkt im gegebenen Radius, wird ein Cluster mit allen gefunden Punkten gebildet.
%Ist die Anzahl Nachbarpunkte kleiner als die Mindestanzahl wird dies als Rauschen markiert.
%Es wird rekursiv mit dem nächsten freien Punkt verfahren, der noch nicht zu einem Cluster gehört und noch nicht bewertet wurde.

%Der DBSCAN-Algorithmus könnte somit verwendet werden, um besondere Merkmale von festzustellen und diese von Rauschen zu unterscheiden.

%Ein Clustering mit DBSCAN zeigte bei uns keine guten Ergebnisse, da wohl die Merkmale

\section{Diskussion}
Die Verwendung des Beschleunigungssensors von einem Smartphone unterscheidet sich zu sehr zwischen einzelnen Modellen und Herstellern.
Hingegen können externe Sensoren an verschiedenen Smartphones genutzt werden, sofern USB-OTG unterstützt wird.
Damit ist eine gleichbleibende Funktionsweise und Erkennungsrate unter verschiedenen Geräten gewährleistet.

Das Filtern und die Signalanalyse muss somit nur für die Skalardaten des Multisensors MPU6050 erfolgen.
Beim Testen der Erkennungsrate wurde in Abb. \ref{fig:FFT_Median} gezeigt, dass die einzelnen Median-Werte der Frequenzgruppen sich weniger stark unterscheiden, als ihre jeweiligen Arithmetischen Mittel.
Dies liegt daran, dass sich in den Daten mehr Ausreißer befinden, die durch Schläge auf eine Tischplatte entstanden.

Die Sensordaten können anschließend als Schlag detektiert und aufgenommen werden.
Dabei spielt das Zeitfenster eine wichtige Rolle für die Genauigkeit und Latenz, wie in Unterabschnitt \ref{sec:zeitfenster} beschrieben.

Desweiteren wurde in Unterabschnitt \ref{sec:zeitfenster} ein Zeitfenster von 550ms gewählt.
Um das ursprüngliche Ziel einer Latenz von <100ms einzuhalten, müsste eine deutlich höhrere Samplerate erreicht werden. Wir vermuten, dass bei einer Samplingrate von ca. 2.5kHz und einem Zeitfenster von 50ms zur Messwertaufnahme eine Latenz von <100ms gelichzeitig zu einer robusten Erkennung erreicht werden könnte.



% Bibliography
\section*{Literaturnachweis}
\bibliographystyle{alpha}
\bibliography{literature}


\end{document}