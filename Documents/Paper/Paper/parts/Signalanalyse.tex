\section*{Signalanalyse}

Eine Möglichkeit um das meiste Rauschen zu unterdrücken ist die Nutzung eines Grenzwertes, um schwache Erschütterungen zu unterdrücken.
Die Sensordaten könnten weiter geglättet werden durch Verwendung eines Box- oder Gauss-Filters.

Anstelle von timbreID, welches laut Entwickler \cite{timbreID} Cluster benutzt und Merkmale sortiert, haben wir den DBSCAN-Algorithmus getetest.

Elbatta und Ashour untersuchen verschiedene Clustering Ansätze und stellen in ihrer Arbeit \cite{Elbatta2013ADM} einen verbesserten DBSCAN-Algorithmus vor.
DBSCAN ist ein  dichtebasierter Cluster-Algorithmus, der als Parameter einen Radius und eine Mindestanzahl an Punkte pro Cluster benötigt.
Im Gegensatz zu partitionierungsbasierten Cluster-Algorithmen, wie zum Beispiel K-Means, besitzt DBSCAN dne Vorteil, dass es freie Formen von Clustern erkennen kann und sich besonders gut für Daten mit Rauschen eignet.
Der Algorithmus startet mit einem zufälligen Punkt.
Befinden sich die Mindestanzahl an Nachbarpunkten um den gewählten Punkt im gegebenen Radius, wird ein Cluster mit allen gefunden Punkten gebildet.
Ist die Anzahl Nachbarpunkte kleiner als die Mindestanzahl wird dies als Rauschen markiert.
Es wird rekursiv mit dem nächsten freien Punkt verfahren, der noch nicht zu einem Cluster gehört und noch nicht bewertet wurde.

Der DBSCAN-Algorithmus könnte somit verwendet werden, um besondere Merkmale von festzustellen und diese von Rauschen zu unterscheiden.

Ein Clustering mit DBSCAN zeigte bei uns keine guten Ergebnisse, da wohl die Merkmale
