\section{Diskussion}

\subsection*{Oberfläche}
\textit{Welche Oberflächen eignen sich am besten?}

Im Laufe des Projektes wurde eine Reihe von Testdurchläufen auf verschiedenen Oberflächen ausgeführt. Dazu wurden 2 Position mit jeweils 10 Schlägen angelernt und anschließend 100 Anschläge druchgeführt. Die Dämpfung erfolgte durch eine flach auf den Tisch gelegte Hand realisiert. Zur Bewertung der Oberfläche wurde die effektive Erkennungsrate herangezogen.

Holztisch, leer | 65%
Holztisch, mit Dämpfern | 68%
Kunststofftisch, leer | 60%
Kunststofftisch mit laufenden PCs | 53%

Erkennbar ist, dass zwischen gedämpfter und ungedämpfert Holztischplatte kaum signifikante Unterschiede bestehen. Die Dämpfung könnte jedoch relevant werden, wenn ein schnelleres Spieltempo angestrebt wird, und Schwingungen aus Anschlägen sich überlagern. In unserem Versuch führten wir den nächsten Anschlag erst aus, wenn die Schwingung der Oberfläche durch den vorherigen Anschlag abgeklungen war.

Weiterhin ist erkennbar, dass die Erkennungsrate bei der Kunststofftischplatte geringer war als bei der Holzplatte.


\subsection*{Zeitfenster}
\textit{Wie groß muss das Zeitfenster für einen Schlag gewählt werden, um eine robuste Erkennung sowie eine geringe Latenz zu erreichen?}



\subsection*{Analyse}
\textit{Welche Informationen können aus den Beschleunigungsdaten gewonnen werden, und welche eignen sich zur Echtzeit-Klassifizierung?}


