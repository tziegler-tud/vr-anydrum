\section{Diskussion}
Die Verwendung des Beschleunigungssensors von einem Smartphone unterscheidet sich zu sehr zwischen einzelnen Modellen und Herstellern.
Hingegen können externe Sensoren an verschiedenen Smartphones genutzt werden, sofern USB-OTG unterstützt wird.
Damit ist eine gleichbleibende Funktionsweise und Erkennungsrate unter verschiedenen Geräten gewährleistet.

Das Filtern und die Signalanalyse muss somit nur für die Skalardaten des Multisensors MPU6050 erfolgen.
Beim Testen der Erkennungsrate wurde in Abb. \ref{fig:FFT_Median} gezeigt, dass die einzelnen Median-Werte der Frequenzgruppen sich weniger stark unterscheiden, als ihre jeweiligen Arithmetischen Mittel.
Dies liegt daran, dass sich in den Daten mehr Ausreißer befinden, die durch Schläge auf eine Tischplatte entstanden.

Die Sensordaten können anschließend als Schlag detektiert und aufgenommen werden.
Dabei spielt das Zeitfenster eine wichtige Rolle für die Genauigkeit und Latenz, wie in Unterabschnitt \ref{sec:zeitfenster} beschrieben.

Desweiteren wurde in Unterabschnitt \ref{sec:zeitfenster} ein Zeitfenster von 550ms gewählt.
Um das ursprüngliche Ziel einer Latenz von <100ms einzuhalten, müsste eine deutlich höhrere Samplerate erreicht werden. Wir vermuten, dass bei einer Samplingrate von ca. 2.5kHz und einem Zeitfenster von 50ms zur Messwertaufnahme eine Latenz von <100ms gelichzeitig zu einer robusten Erkennung erreicht werden könnte.

