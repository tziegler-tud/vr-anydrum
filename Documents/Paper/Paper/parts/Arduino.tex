\section{Arduino}

\subsection*{Arduino Mikrocontroller}
Für die Umsetzung wurde ein "Arduino Nano" Mikrocontroller mit Prozessor ATmega328P verwendet. Dieser führt selbst keine Datenverarbeitung aus, sondern bietet lediglich die Kommunikationsschnittstelle zwischen Smartphone und Sensoren. Die Kommunikation zum Smartphone erfolgt über die Serielle Schnittstelle, die Kommunikation zum Sensor über den I$^2$C-Bus. 


\subsection*{CH340}
Bei dem Arduino Chip CH340 handelt es sich nach dem Datenblatt \cite{CH340} um einen BUS Konvertierungs Chip, der eine serielle USB Schnittstelle ansprechen kann.
Es unterstützt sowohl 5V, als auch 3.3V Spannungsquellen zum Betrieb.
Die Baudraten reichen von 50bps bis zu 2Mbps.


\subsection*{MPU-6050}
Nach dem Datenblatt \cite{MPU6050} besitzt der MPU-6050 sensor ein Gyroskop mit den X-, Y- und Z-Achsen.
