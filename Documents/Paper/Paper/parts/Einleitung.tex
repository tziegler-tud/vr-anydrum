\section*{Einleitung}
Diese Projektarbeit entstand im Rahmen des Modules "Virtuelle Realität" im Sommersemeter 2018. Die Zeilstellung war die Entwicklung eines Systems, bestehend aus Hard- und Software, mithilfe dessen eine beliebige Oberfläche als Schlagzeug verwendet werden kann.
Eine angeschlagene Position soll dafür in Echtzeit klasifiziert und einem der vorher erlernten Schlaginstrumente zugeordnet werden.

Die Vergabe der Projektthemen und erste Besprechung dieser erfolgte am 16.04.2018. Am 28.05. erfolgte eine Zwischenpräsentation, im Rahmen dieser die bisherigen Konzepte der verschiedenen Teams gegenseitig vorgestellt und diskutiert wurden. 
Die Abgabe und abschließende Präsentation der Ergebnisse erfolgte am 09.07.2018. 
Der Zeitraum der Bearbeitung erstreckte sich damit über 12 Wochen.

Im Rahmen einer ersten konzeption wurden folgende Maßgaben für die Umsetzung aufgestellt:
\begin{itemize}
	\item Das entstandene System soll portabel, einfach anzuwenden und wenig spezielle Hardware erfordern
	\item Um ein realistisches Spielgefühl zu erzeugen, soll die maximale Latenz zwischen Anschlag und Abspielen des Sounds 100ms betragen
	\item Das System soll robust funktionieren und dem Anwender ein Feedback über die Zuverlässigkeit der Erkennung liefern, sodass dieser eventuell Verbesserungsmaßnahmen durchführen kann (z.Bsp. Reduzierung von Hintergrundrauschen)
\end{itemize}

Um die Realisierung der Projektidee in diesem zeitrahmen möglich zu machen, wurden zu Beginn einige Vereinbarungen getroffen:
\begin{itemize}
	\item Als Oberfläche sollen vorerst nur Tische mit Holz- oder Kunststoffplatten verwendet werden
	\item Maximal sollen drei verschiedene Positionen unterschieden werden
	\item Im Sinne einer Prototypentwicklung sollen Funktionalität und Beweis der Konzeptplausibilität höhere Priorität als Benutzerfreundlichkeit haben
    \item In diesem Projekt soll die Eigenschwingung der angeschlagenen Oberfläche zur Auswertung genutzt werden. 
\end{itemize}


