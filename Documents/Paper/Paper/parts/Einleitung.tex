\section{Einleitung}
Mit Hilfe von Piezosensoren, Kraftsensoren, soll es möglich sein von beliebigen Oberflächen Schwingungen aufzunehmen, die beim Klopfen auf das Objekt entstehen.

Das Klopfen soll abhängig von der Position auf dem Objekt und der Stärke einen unterschiedlichen DrumSound erzeugen. 
Zu den Möglichen Geräuschen zählen Trommeln, Basstrommel und Zimbal.

Die Trommelgeräusche werden mittels PureData synthetisiert. 
Die Geräusche werden in Abhängigkeit von der Entfernung und Stärke zum Piezosensor ausgewählt.
Eine Mobile-App wertet die Daten des Piezosensors aus und sppielt das entsprechende Geräusch ab.

