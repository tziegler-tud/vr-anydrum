\section{Konzept}
\label{sec:Konzept}
Ein naiver Ansatz wäre, mehrere Sensoren auf der Oberfläche zu verteilen und eine ein- oder zweidimensionale Lokalisation mittels Bestimmung der Laufzeitunterschiede durchzuführen.
Dies setzt allerdings die Verwendung mindestens zweier (eindimensional, max. 3 Positionen) bzw. dreier (zweidimensional) Sensoren. 
Die Verwendung mehrerer Sensoren bedeutet jedoch auch steigende Anforderungen an die verarbeitende Hardware, höherer Realisierungsaufwand, höhere Anforderungen an den Anwender und nicht zuletzt höheren Hardwareaufwand und -abhängigkeit, was die Portabilität und Verwendbarbarkeit deutlich einschränkt.

Daher soll lediglich ein Sensor verwendet werden. Dies bedeutet jedoch, dass die Unterscheidung der Anschlagpositionen lediglich anhand der Schwingungscharakteristiken erfolgen muss.

Wird eine Tischplatte an einer Position angeschlagen, führt diese eine Eigenschwingung aus.
Je nach Material, Lagerung, Obejekte auf der Oberfläche, etc., unterscheiden sich verschiedene Tischplatten in ihrer Impulsantwort.
Außerdem unterscheidet sich die ausgeführte Schwingung je nach Position und Art der Erregung.
Ein Schlag mit den Fingerknochen erzeugt eine andere Schwingung, als ein Schlag mit der flachen Hand.
Dies lässt sich anhand des entstehenden Geräusches verifizieren.
Ebenso erzeugt ein Schlag auf die Tischkante eine andere Schwingung, als ein Schlag in die Tischmitte, welches sich vor allem in der unterschiedlichen Nachhallzeit zeigt.

Gewisse charakteristische Eigenschaften sind bauartbedingt durch den Tisch vorgegeben. 
Diese können vom Anwender nicht beeinflusst werden, folglich muss die Software in der Lage sein, diese Unterschiede in der Verarbeitung zu berücksichtigen. 
Trotzdem hat der Anwender gewisse Möglichkeiten, die Schwingung der Tischplatte zu beeinflussen, beispielsweise durch Anbringen von Massen, Dämpfern, etc.
Dies kann und soll bewusst eingesetzt werden, um die Schwingungscharakteristiken der Anschlagspositionen möglichst unterschiedlich zu gestalten.  

Um eine hohe Flexibilität in der Anwendung zu liefern, soll auf die Verwednung eines PCs/Laptops verzichtet werden. Stattdessen soll die softwareseitige Umsetzung mittels eines Smartphones erfolgen.

Zur Aufnahme der Schwingung wurden verschiedene Möglichkeiten erprobt:
\begin{itemize}
	\item Die Verwendung eines Piezzosensors zur Aufnahme von Schwingungen.
	\item Messen der erzeugten Beschleunigung, mit Hilfe eines internen Beschleunigungssensors des Smartphones
	\item Das Messen mittels externer Beschleunigungssensoren eines Arduino Mikrocontrollers  
\end{itemize}

