\section{Ausblick}

Eine Möglichkeit um Ausreiser genauer zu erkennen bieten Clustering-Verfahren.
Elbatta und Ashour untersuchen verschiedene Clustering Ansätze und stellen in ihrer Arbeit \cite{Elbatta2013ADM} einen verbesserten DBSCAN-Algorithmus vor.
DBSCAN ist ein  dichtebasierter Cluster-Algorithmus, der als Parameter einen Radius und eine Mindestanzahl an Punkte pro Cluster benötigt.
Im Gegensatz zu partitionierungsbasierten Cluster-Algorithmen, wie zum Beispiel K-Means, besitzt DBSCAN den Vorteil, dass es freie Formen von Clustern erkennen kann und sich besonders gut für Daten mit Rauschen eignet.
Der Algorithmus startet mit einem zufälligen Punkt.
Befinden sich die Mindestanzahl an Nachbarpunkten um den gewählten Punkt im gegebenen Radius, wird ein Cluster mit allen gefunden Punkten gebildet.
Ist die Anzahl Nachbarpunkte kleiner als die Mindestanzahl wird dies als Rauschen markiert.
Es wird rekursiv mit dem nächsten freien Punkt verfahren, der noch nicht zu einem Cluster gehört und noch nicht bewertet wurde.

Bei der Verwendung eines solchen Verfahrens spielt jedoch die Vorverarbeitung der Daten eine wichtige Rolle.
Ebenso muss das Zeitfenster für die Daten mitbeachtet werden.

Um die Latenz zu verringern, könnten ...
