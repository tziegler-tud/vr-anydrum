\section{Umsetzung}
Für die Umsetzung wurde ein Android-Smartphone, ein Lenovo K6, mit der Android-Version 7.0 verwendet.
Hingegen zu älteren Smartphones unterstützen neuere den USB On-The-Go (OTG) Modus, womit die Stromversorgung des Arduinos über USB-Anschluss gewährleistet wird.

Desweiteren wurde ein Arduino CH340 chip verwendet und als Beschleunigungssensoren das Modell MPU-6050.
Die Android-App wurde in Java geschrieben.
Zur Kommunikation mit dem Arduino und abgreifen der Beschleunigungssensoren des Sensors wurden die Bibliotheken \textit{usb-serial-for-android} \cite{mik3y} und \textit{UsbSerial} \cite{felHR85} verwendet.

Die synthetisch erzeugten Schlagzeuggeräusche für Hihat, Snare und Bass wurden mit PureData \cite{puredata} erzeugt. 
Zur Nutzung der PureData-Patches auf dem Android Gerät, wurde die Bibliothek \textit{pd-for-android} \cite{pdAndroid} von \textit{libpd} verwendet.
Diese unterstützt keine erweiterten Funktionen, wie sie in PureData Extended vorkommen.

